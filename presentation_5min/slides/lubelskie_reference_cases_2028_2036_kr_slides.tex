\documentclass[11pt,a4paper]{article}
\usepackage{fontspec}
\setmainfont{Malgun Gothic}
\usepackage{geometry}
\geometry{a4paper, landscape, margin=12mm}
\pagestyle{empty}
\setlength{\parskip}{0.35em}
\setlength{\parindent}{0pt}

\newcommand{\slide}[2]{
  \clearpage
  \begin{center}
    \LARGE\bfseries #1\par
    \vspace{0.2cm}
  \end{center}
  #2
}

\begin{document}

\slide{ERDF/JTF 참고사례: Lubelskie 대상 프로젝트 벤치마크}{
목표: 발표에서 ERDF+JTF 조합 예산/구조를 설명할 때 바로 참조할 수 있는 정합 사례 정리
}

\slide{ERDF 사례 2건(인프라·접근성 계열)}{
\textbf{ERDF-1: S19 Lublin–Lubartów 북부 구간}
\begin{itemize}
  \item 총사업비 1,394.8m PLN / EU 지원 612.1m PLN
  \item 기간 2021–2028
  \item 메시지: 교통 접근성 개선이 지역 기초수요를 끌어올리는 대표형
\end{itemize}

\textbf{ERDF-2: Lublin 통합 대중교통 확장}
\begin{itemize}
  \item EU 지원 190.7m PLN
  \item 기간 2023–2029
  \item 메시지: 농촌·외곽 접근권 개선 및 이동권 보완
\end{itemize}
}

\slide{JTF 사례 2건(노동시장 전환 계열)}{
\textbf{JTF-1: Konin 재교육 — "Droga do zatrudnienia po węglu"}
\begin{itemize}
  \item EU 지원 180.0m PLN
  \item 기간 2023–2029
  \item 메시지: 전환 직종 종사자 재교육과 고용연결의 실무형 모델
\end{itemize}

\textbf{JTF-2: Śląskie 제로에미션 이동성 전환}
\begin{itemize}
  \item EU 지원 132.0m PLN
  \item 기간 2025–2029
  \item 메시지: 산업 거점 전환에서 고용안전망을 동반한 전환완충 모델
\end{itemize}
}

\slide{Lubelskie 제안(2028–2036) 반영 포인트}{
\textbf{ERDF 반영}
\begin{itemize}
  \item 우선순위: 인프라(교통·에너지), 접근성, 저탄소 기반 시설
  \item 산정값(안): 건별 ERDF 28–57m EUR
\end{itemize}

\textbf{JTF 반영}
\begin{itemize}
  \item 우선순위: 전환 취약군(저소득 가구, 유지보수·운행·시설관리 종사자) 고용 전환
  \item 산정값(안): 건별 JTF 14–26m EUR
\end{itemize}

\textbf{핵심 메시지}
\begin{itemize}
  \item ERDF로 기반을 만들고 JTF로 충격을 흡수하는 구조가 발표 적합
\end{itemize}
}

\end{document}